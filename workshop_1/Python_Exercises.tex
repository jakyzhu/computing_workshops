\documentclass{article}
\usepackage{enumerate, amsmath ,amsfonts}
\usepackage[margin=1in]{geometry} %change page margins
\usepackage[english]{SASnRdisplay} %codeblocks in SAS and R. English option gives English captions (otherwise they'll be in Danish)
\usepackage{hyperref, graphicx, hyperref} %embed hyperlinks in the pdf
\usepackage{times}

\title{Introduction to Python\\
	\Large{Exercise}}
\author{Adam Peterson}

\begin{document}
	\maketitle

\section{Rock Paper Lizard Spock Game}

	\begin{figure}[h]
		\centering
		\includegraphics[width=.7\textwidth]{rpls.jpg}
		\caption{Helpful Diagram}
		\label{fig: RPLS}
	\end{figure}
	
	\subsection{Directions}
	\begin{enumerate}
		\item  Go to the github repository that holds this presentation material, linked \href{https://github.com/apeterson91/computing_workshops/tree/master/workshop_1}{Github Link}
		\item Use the template to create a game that simulates the the Rock Paper Lizard Spock Game detailed above
		\item There are two helper functions you have to write, they're fairly self explanatory
		\item The key \textit{challenge} behind simulating this game is actually understanding the mathematical pattern to the game - think of modulus relations to help you solve this
		\item if you need help Ask! 
	\end{enumerate}




\end{document}